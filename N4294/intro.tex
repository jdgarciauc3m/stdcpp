\section{Introduction}

This paper revisits the proposal from N3875~\cite{n3875} and its associated
problems and tries to find a solution to run-time bounds arrays based on
\emph{array constructors}. The final goal is to provide a generalized way of
defining automatic storage arrays (i.e. stack allocated or side-stack allocated)
whose size is not known at compile time.

\section{Problem}

\emph{Variable Length Arrays} (VLAs) have existed in C since C99~\cite{c99}. However,
the introduction of VLAs in C++ standard as they are, has been identified as
problematic.

Several attempts have been made to introduce a similar facility in the C++
programming language. A solution was incorporated in the working draft of C++14
in the Bristol meeting (N3691~\cite{n3691}). At the Chicago meeting the committee
decided that this solution was controversial and this part was removed from the
working draft (N3797~\cite{n3797}). The committee decided that a separate
technical report on \emph{array extensions} should address this issue.

Previous attempts to run-time sized arrays include: run-time sized arrays with
automatic storage, \cppid{dynarray}, and \cppid{bs\_array}. A summary of such ideas can be found
in N3875~\cite{n3875}.

