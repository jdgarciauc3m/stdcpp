\section{Open issues}

In this section we discuss open issues that need to be solved regarding this proposal.

\subsection{Alternate constructors}

Imposing the (transitive) requirement that constructors need to be inlined may
be seen as very restrictive. We think that the most common use of this feature
is defining automatic storage objects at function scope level. Thus this
restriction may be not so critical.

However, an alternate approach could be allowing that a constructor body is
first inlined and the its body is refined.

\begin{lstlisting}
class vec {
public:
  vec(int n) : v[n]{} {} // inline definition
private:
  double[] v;
};

// Definition somehwere else
vec::vec(int n) // Member initiation not allowed here
{
  do_some_addtional_thing();
}
\end{lstlisting}

In this case the inline definition can be used for computing the size of the
object at the construction call site. However, after that the constructor body
(if one is existing) may be called. For the sake of avoiding redundancy in this
case, we do not allow to repeat the member initiation list.

\subsection{Size determination}

The current proposal implies that the size of any \emph{run-time bound array
data member} can be derived from the environment where the inline constructor is
defined.

In particular, this proposal does not allow the following example (slightly
modified from an example provided by Lawrence Crowl).

\begin{lstlisting}
// a.h
struct A {
  bs_array<double> storage;
  A(int n);
};

// a.cc
  extern int config;
  A::A(int n) : storage(n*config) {}
\end{lstlisting}

As \verb+struct A+ has a data member which is a \emph{run-time size bound class}
(the \verb+bs_array+), it is considered itself to be also a \emph{run-time size
bound class} and the same restrictions apply to it. Thus, it needs that its
constructor is defined inline.

In this proposal we have not addressed this problem. We think it should be first
clarified if this use case is important enough to be solved.

Possible solutions are:

\begin{itemize}

\item Require that the variable (i.e. \verb+config+) is visible in the point of definition of the
inline constructor.

\item Modify the proposal so that it is not required that the constructor is
defined inline.

\end{itemize}

\subsection{Multi-dimensional data members}

This proposal initially tries to support single-dimension \emph{run-time bound}
data members. However, if needed it could be extended to support multiple
dimension \emph{run-time bound} data members.

\begin{lstlisting}
class stack_matrix {
public:
  stack_matrix(int r, int c) : v[r][c]{} {}
  // ...
private:
  double[][] v;
};
\end{lstlisting}

