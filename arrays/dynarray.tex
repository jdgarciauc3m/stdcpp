\subsection{\emph{dynarray}}

A class providing a dynamic array was proposed in N3532~\cite{n3532} and adopted
into the working draft as N3662~\cite{n3662}. However, this was not first time
such a class was proposed dating back to N2648~\cite{n2648} in 2008.

While the proposal claims that a \emph{pure-library} implementation is possible,
such an implementation would be probably based on dynamic memory allocation. To
provide stack allocation of automatic variables compiler magic is needed by
giving the compiler the freedom of implementing construction and destruction of
\emph{dynarray} directly.

While the border between the library and the compiler has been crossed before
(e.g. \verb+std::initializer_list+) we believe alternate solutions can be used
here. 

In any case, it is true that \verb+dynarray+ solves the interface problems that
run-time sized arrays would be promoting and better code can be written.

\begin{lstlisting}
string g(const dynarray<int> & v) {
  std::string result;
  for (auto x : v) {
    result += std::to_string(x);
  }
  return result;
}

string f(int n) {
  dynarray<int> v; // Where is this allocated?
  std::iota(v.begin(), v.end(), 1);
  return g(v);
} 
\end{lstlisting}

However, a more serious concern is that the user has no guarantee whether
a \verb+dynarray+ object will allocated in automatic or dynamic storage.
\textbf{This is specially important for applications where performance 
needs to be predictable}.
