\subsection{Sized constructors}

Restricting constructors of \emph{run-time size bound classes} to be inline have
been seen as problematic. We have outlined some concerns about this issue in the
previous section.

Here we outline an alternative where the size of run-time bound array data
members is part of the constructor specification. This proposal is based in
(what we think is) a simplification over an original idea proposed by Daveed
Vandevoorde.

\subsubsection{Constructors with size specification}

A \emph{run-time size bound class} needs to specify the size of its array data
members upon construction. We propose to make such specification part of the
constructor declaration.

For this purpose we propose to add to the constructor syntax a
\emph{sizeof-specifier}.

\begin{lstlisting}
class my_vec {
public:
  my_vec(int n) sizeof(v[n], w[2*n]);
  my_vec() sizeof(v[2], w[4]);
private:
  int z;
  double[] v;
  int[] w;
};

my_vec::my_vec(int n) : z{n} {}

my_vec::my_vec() : z{0} {}
\end{lstlisting}

We think this is simpler than the storage specification suggested by Daveed
Vandevoorde and quoted in N3810. With that syntax the above example would be:

\begin{lstlisting}
class my_vec {
public:
  my_vec(int n) double storage_v[n], int storage_w[2*n];
  my_vec()  double storage_v[2], int storage_w[4];
private:
  int z;
  double * v;
  int * w;
};

my_vec::my_vec(int n) : z{n}, v{storage_v}, w{storage_w} {}

my_vec::my_vec() : z{0}, v{storage_v}, w{storage_w}{}
\end{lstlisting}

We see storage specification option to look more verbose. Besides, it implies to
manage two names per array (one for the storage and another one for the pointer).

\subsubsection{Implementing \emph{bs\_array}}

An implementation of \verb+bs_array+ is possible with sized constructors.

\begin{lstlisting}
template <class T>
class bs_array {
public:
  using value type = T;

  bs_array(int n) sizeof(v[n]);
  bs array(const bs array & a) = default;
  bs array(bs array &&) = delete;

  // Unchecked access
  T & operator[](int i) { return v[i]; }
  const T & operator[](int i) const { return v[i]; }

  // Range checked access
  T & at(int i );
  const T & at(int i) cosnt;

  // begin/end
  T * begin() { return v; }
  const T * begin() const { return v; }
  T * end() { return v+sz; }
  const T * end() { return v+sz; }

  int size () const { return sz; }
  T * data() { return v; }
private:
  int sz;
  T[] v;
};
\end{lstlisting}

\subsubsection{Initialization options}

Initialization for \emph{run-time bound array data members} is performed at the
constructor through initialization lists as it would have been done for any
other array data member.

\begin{lstlisting}
class vec {
public:
  vec(int n) sizeof(v[n], w[n]);
private:
  double[] v;
  point[] w;
};

vec::vec(int n) : v{}, w{} {} // v[i] == 0.0, w[i] == point{}
\end{lstlisting}

It is possible to \emph{default-initialize} any array data-member by not
including it in the initialization list.

\begin{lstlisting}
vec::vec(int n) {} // v and w are default-initialized
\end{lstlisting}

It is also possible to provide \emph{initializer-lists} for performing array
initialization. If the list provides less initializers than the array data
member size, the rest of elements are \emph{value-initialized}. If the list
provides more elements than the array data member size exceeding elements are
discarded.

\begin{lstlisting}
class myvec {
public:
  myvec() sizeof(v[4], w[2]);
  myvec(int n) : sizeof(v[n], w[n]);
private:
  double[] v;
  float[] w;
};

myvec() :
  v{1.0, 2.0}, // {1.0, 2.0, 0.0, 0.0}
  w{1.5, 2.5, 3.5} // {1.5, 2.5}. Third value (3.5) is discarded
{}

myvec{int n) :
  v{1.0, 2.0 }, // Discarding if n<2
  w{1.5, 2.5, 3.5 }, // Discarding if n<3
{}
\end{lstlisting}

\subsubsection{Run-time size bound objects}

If a class has a data member which is of a \emph{run-time size bound class} it
becomes itself a \emph{run-time size bound class}. In that case we require that
the call to the constructor of such sub-object is specified in the \emph{sizeof}
specifier of the constructor.

\begin{lstlisting}
class A {
public:
  A(int n) sizeof(v[n]);
private:
  double[] v;
};

class B {
public:
  B() sizeof(a{4});
  B(int m) sizeof(a{m});
private:
  A a; // A is run-time size bound -> B is run-time size bound
};

B::B() {}
B::B(int m) {}
\end{lstlisting}

This is much less restrictive that requiring the inlining of the full
constructor. This requirement allows for size determination from the call site.

A similar restriction applies to inheritance:

\begin{lstlisting}
class A {
public:
  A(int n) sizeof(v[n]);
private:
  double[] v;
};

class B : public A { // A is run-time size bound -> B is run-time size bound
public:
  B() sizeof(A{3});
  B(int m) sizeof(A{m});
};

B::B() {}
B::B(int m) {}
\end{lstlisting}

\subsubsection{Contexts of use}

As with inline constructors alternative, we propose that the only use of a
\emph{run-time size bound class} object is as an automatic variable.

\subsubsection{\emph{sizeof}}

As with inline constructors alternative, we do not plan to support operator
\verb+sizeof+.

\subsubsection{Non-inlined constructors}

In this option there is no requirement to force constructors to be
\emph{inline}. This allows constructors to be defined elsewhere and to have
complex bodies when needed.

\begin{lstlisting}
class vec {
public:
  vec(int n) sizeof(v[n]);
private:
  double[] v;
};

// Definition somewhere else
vec::vec(int n) : v{}
{
  do_some_additional_thing();
}
\end{lstlisting}

\subsubsection{Size determination}

This proposal does not allow writing the example from Lawrence Crowl already cited
in Section~\ref{sec:inline-size-det}.

\subsubsection{Multi-dimensional data members}

Multi-dimensional data members can also be expressed with this alternative.

\begin{lstlisting}
class stack_matrix {
public:
  stack_matrix(int r, int c) sizeof(v[r][c]);
private:
  double[][] v;
};
\end{lstlisting}
