\subsection{Runtime-sized arrays with atuomatic storage}

As stated by Jens Maurer in \emph{Runtime-sized arrays with automatic storage
duration} series (N3366~\cite{n3366}, N3412~\cite{n3412}, N3467~\cite{n3467},
N3497~\cite{n3497}, N3639~\cite{n3639}) there are situations where a \emph{user wishes to
allocate a local array whose size is not known at compile-time, but at runtime
only}.

This kind of uses can be be handled by a \verb+std::vector+ usage. However, such
approach may lead to a serious performance penaly derived from dynamic memory
allocation and deallocation

\begin{lstlisting}
string f(int n) {
  std::vector<int> v(n); // std::vector allocation
  std::iota(v.begin(), v.end(), 1);
  std::string result;
  for (auto x : v) {
    result += std::to_string(x);
  }
  return result;
} // std::vector deallocation
\end{lstlisting}

The use of a run-time sized array with automatic storage allows to avoid dynamic
memory allocation by using automatic storage.

\begin{lstlisting}
string f(int n) {
  int v[n];
  std::iota(v, v + n, 1);
  std::string result;
  for (auto x : v) {
    result += std::to_string(x);
  }
  return result;
} 
\end{lstlisting}

This approach promotes the use of \emph{pointer-plus-size} interface when arrays
are passed around.

\begin{lstlisting}
string g(int * v, std::size_t n) {
  std::string result;
  for (std::size_t i=0; i!=n; ++i) {
    result += std::to_string(v[i]);
  }
  return result;
}

string f(int n) {
  int v[n];
  std::iota(v, v + n, 1);
  return g(v, n);
} 
\end{lstlisting}

Stroustrup lists in N3810~\cite{n3810} reasons why that interface promotion is not desirable
including:

\begin{itemize}

\item The number of elements of a run-time sized array is not available and the
user needs to \emph{remember} it.

\item The type used to access elements can implicitly change after the array has
decayed to a pointer.

\end{itemize}
