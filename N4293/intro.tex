\section{Introduction}

Contracts are part of the interface of an operation. They state under which
circumstances an operation is guaranteed to work and what conditions are
ensured after execution. Additionally, they may establish invariant conditions
that are guaranteed to be true for values of a given type.

In summary contracts define:

\begin{itemize}
\item Conditions that must be met prior to an operation execution.
\item Conditions that the operation guarantees after an operation execution.
\item Conditions that a value from a type must hold and define a valid state.
\end{itemize}

During the \emph{Urbana-Champaign} meeting the \emph{Evolution Working Group} discussed
general directions for contracts programming in C++. As part of this discussion
some general questions were addressed:

\begin{enumerate}
\item Do we want contracts?
\item Is performance an important feature of contracts?
\item Is correctness an important feature of contracts?
\item Should contracts be in declarations?
\item Should build modes be standardized?
\end{enumerate}

While, there are many other important questions, these questions are used to
drive this paper and as an starting point for a contracts programming design.
This paper also revises ideas introduced in N4110~\cite{n4110} at the light of
the answers from the committee to those questions.

\subsection{Contracts programming for C++}

The current situation regarding contracts is that they conceptually exist in
libraries (including the standard library). Some contracts are documented while
others simply are not documented at all. In terms of contract violations
handling in some cases an exception is thrown while in other cases it leads to
undefined behavior.

As an example, in the standard library some operations define a wide contract
(those having a well defined behavior for all calls), even though under certain
argument values they throw an exception (e.g. \cppid{out\_of\_range}).

The standard library also contains operations defining a narrow contract (those
having a well defined behavior for a subset of all calls). Violation of such
contracts leads to undefined behavior.

It is important to stress that contracts include not only preconditions, but
also postconditions, and invariants. Furthermore, the latter are as important as
preconditions or even more.

A key difference between wide and narrow contracts is that while wide contracts
are always checked, narrow contracts are not necessarily checked.

\ewgdirection{There was a great agreement that C++ should have support for
contracts programming.}

\subsection{Contracts and performance/correctness}

The basic goal of contracts should be to improve correctness. That is to verify
that:

\begin{itemize}
\item Preconditions are met before operation calls.
\item Postconditions are met met after operation execution.
\item Invariants are held.
\end{itemize}

Besides, the use of contracts has additional effects beyond correctness: better
diagnostics of programming errors, check elision, better reasoning about
programs, and potential use by external tools.


\ewgdirection{There was an agreement that both features are desirable for contracts
programming. The key identified feature of contracts was correctness. However,
it was agreed that performance is also an important feature.}

\subsection{Contracts and declarations}

Any of the proposed initial designs on contracts require language support.
Basically, there are two approaches for specifying contracts:

\begin{itemize}
\item A \textbf{library} approach with language support.
\item A \textbf{language} approach with contracts being part of functions
specifications.
\end{itemize}

A \textbf{library} approach (with language support) expresses contract
assertions in the body of the functions. This makes contracts easy to deploy.
However, contract validation is hidden into function bodies what makes much
harder to compilers and other tools to reason about program behavior.	

A \textbf{language} approach requires more profound language changes. And
advantage of this approach is that contract validation may be part of the
function interface instead of the function body. This approach makes easier for
compilers and other tools to reason about programs

\ewgdirection{There was a great support to specifying contracts in declarations. While
contracts in the body were also discussed the committee did not get to any final
voting on this second issue.}

\subsection{Standardized build modes}

Build modes have never been part of the C++ standard. However, almost every
implementation includes such a concept. Furthermore, the concept of a
\emph{debug} mode is hidden behind the \cppid{NDEBUG} macro.

One option for contracts programming is to define a set of build modes. For
example N4135~\cite{n4135} proposes four build modes: \emph{none}, \emph{opt},
\emph{debug}, and \emph{safe}. Another option is to avoid bringing build modes
into the standard. This approach requires to specify the semantics for the
single mode in the standard, while other modes are possible but outside the
standard specification.

\ewgdirection{There was a consensus that build modes should not be
standardized.}

\subsection{Summary of directions}

Given this, the directions for contracts programming can be summarized as
follows:

\begin{itemize}
\item C++ should have some form of support for contract programming.
\item C++ contracts main goal is to support program correctness.
\item C++ contracts should also have performance into consideration.
\item C++ should support contracts in function declarations.
\item C++ should not standardize build modes.
\end{itemize}
