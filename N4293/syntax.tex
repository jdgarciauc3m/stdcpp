\section{Syntax}

\subsection{Preconditions}

A function precondition is part of its declaration:

\begin{lstlisting}
double square_root(double x)
  expects{x>=0};
\end{lstlisting}

When an operation has multiple preconditions they can be combined with boolean operators.

\begin{lstlisting}
class my_vector {
//...
  double get_at(int i)
    expects{i>=0 && i<size && vec!=nullptr};
//...
private:
  double * vec;
  int size;
};
\end{lstlisting}

The \cppkey{expects} clause may contain any expression that is contextually
convertible to \cppkey{bool}. However, no side effects should be allowed in such
predicates.

\begin{lstlisting}
void f(int i) expects{i}; // OK
void g(int i) expects{i++}; // Ill-formed
\end{lstlisting}

In particular, an expression in a precondition may invoke functions as long as
they do not have side effects.

\begin{lstlisting}
class my_vector {
//...
  double get_at(int i)
    expects{ valid_index(i) && valid_vec() };
//...
  bool valid_index(i) const { return i>=0 && i<size; }
  bool valid_vec() const { return vec!=nullptr; }
private:
  double * vec;
  int size;
};
\end{lstlisting}

\subsection{Postconditions}

Any operation specification may have in its declaration a list of
postconditions that the developer guarantees to be held after the operation
execution.

\begin{lstlisting}
class string {
  //...
  void reserve(size_type res_arg = 0)
    expects{res_arg < max_size()}
    ensures{capacity() >= res_arg};
  //...
};
\end{lstlisting}

Expressions in a postcondition follow the same rules than expression in preconditions.

In the context of a postcondition the function name refers to the returned
value of the function.

\begin{lstlisting}
double square_root(double x)
  expects{x>=0.0}
  ensures{square_root>=0.0};
\end{lstlisting}

When a value may change during function execution previous value may referred
through operator \cppid{pre}.

\begin{lstlisting}
double increment(int & x)
  ensures{x == pre(x) + 1};
\end{lstlisting}

Operator \cppkey{pre} can also be applied to \cppid{*this}.

\begin{lstlisting}
class point {
  //...
  void move(double deltax, double deltay)
    ensures{x == pre(x) + deltax && y = pre(y) + deltay};
\end{lstlisting}

