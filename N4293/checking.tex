\section{Contract checking}

There is a tension on to which extent contracts should be checked. A contract
design should be able to accommodate different user group needs.

On one, hand there are many situations where a contract can be proved to be 
satisfied and consequently the associated checks are unnecessary.

As an example, in the following fragment, it can be proved that the expectation
of \cppid{i<size} en ensured to be true and no check is needed.

\begin{lstlisting}
vector v(100); // ensures size==100
for (int i=0;i<100;++i) {
  v[i] = f(i); // expects i<size
}
\end{lstlisting}

Besides, if a contract can be proved to be not satisfied at all the program
can me made ill-formed.

\begin{lstlisting}
vector v(10); // ensures size==10
for (int i=0;i<100;++i) {
  v[i] = f(i); // expects i<size -> Ill-formed
}
\end{lstlisting}

On the other hand, there are situations where a contract cannot be proved at
compile-time In that case there are basically two options: either do not
perform any checking at all, or perform some degree of checking.

Performing no checking at all favors performance. This is what it is currently
done in many standard library operations with a \emph{narrow contract}. However,
we get this at the price of library undefined behavior.

Performing checks at run-time favors correctness. This is what many debug
implementations of the standard library do.  However, we get this additional
correctness at the price of less performance.

A solution to this tension is to make run-time checks for contracts 
\emph{implementation-defined}. Thus, every implementation should define 
which checks are performed at run-time.

In summary, any implementation should offer both:

\begin{itemize}

\item A mode where no run-time check at all is performed.

\item One or more modes with different degrees of checking at run-time.

\end{itemize}

In the rest of this section, options and constraints for implementations are explored.

\subsection{Non-checked mode}

Even in a non-checked mode an implementation is 
still required to perform checks at compile-time.

It can be argued that allowing an implementation not to perform any run-time
checks could lead to undesired undefined behaviors. For example:

\begin{lstlisting}
void f(int i) {
  myvector v{20, 0.0}; // A vec of 20 doubles initialized to 0
  v[2] = 1.0;  // Not checked. OK.
  v[99] = 2.0; // Ill-formed. Contract violation.
  v[i] = 3.0;  // Could be undefined behavior if i>=20
\end{lstlisting}

However, this is not worse than the current situation with language arrays and
the \cppid{std::vector} type.

\subsection{Checked modes}

A \emph{checked mode} performs checks for contracts that cannot be
proved at compile-time.

\begin{lstlisting}
void f(int i) {
  myvector v{20, 0.0}; // A vec of 20 doubles initialized to 0
  v[2] = 1.0;  // Check elided.
  v[99] = 2.0; // Ill-formed. Contract violation.
  v[i] = 3.0;  // Run-time contract violation, if i>=20
}
\end{lstlisting}

Contract checking is a property of the client code. This means that an
implementation could eventually allow two translation units to use some
function where one translation unit performs checking and a different
translation unit does not perform that checking.

\begin{lstlisting}
// libf.h
void f(int i) expects{i>0};

// ej1.cpp with checking
int g(int i) {
  f(i); // Will check i>0
  // ...
}

// ej2.cpp without checking
int h(int i) {
  f(i); // No checking
  // ...
} 
\end{lstlisting}

\subsection{Multiple checking modes}

This paper makes a distinction between a build mode and a checking mode. A build
mode defines the options used to build a program. While the standard does not
establish any build mode at all, most vendors support multiple build modes.
Typically those include a \emph{release} mode and a \emph{debug} mode. \textbf{No
standardization is proposed here regarding build modes}.

On the other hand, a \emph{checking mode} defines which level of checking (if any) is
active for a specific translation unit.

One approach used in other languages and approaches is to define checking modes
in terms of which component of the contract is checked.
Those modes could be:

\begin{itemize}
  \item \textbf{none}: No checking performed at all.
  \item \textbf{pre}: Only precondition are checked.
  \item \textbf{post}: Both preconditions and postconditions are checked.
\end{itemize}

Another approach is to define several checking levels, and specify for every
check its level. This level could be numeric, so that any check with a level
lower than a threshold is performed. This is also compatible to the levels
provided in \cite{n4135} with a small set of levels. 

As an example, this levels could be:

\begin{itemize}
  \item \textbf{rel}: No checking performed at all.
  \item \textbf{min}: Only critical checks are performed.
  \item \textbf{dbg}: Most checks are performed. Suitable for debugging.
  \item \textbf{safe}: All checks are performed, even the most costly ones.
\end{itemize}

It is important to remark that the checking mode affects to the client code when
it is compiled. Different translation units may be compiled with different
checking modes.

In any case, how the level of a contract is specified needs further exploration 
and is left in this paper as an open point.

\subsection{The effects of a broken contract}

A key element for contracts success is which should be the effect of breaking a
contract in a \emph{checked mode}. 

Set of principles:

\begin{itemize}

\item Contract checking is a property of client code. Clients must be able to
make the choice of how much (if any) checking they want to use.

\item A mode is needed where no run-time overhead is added over a non contract
based version. Still that mode, may provide diagnostics for contracts that can
be proved broken during translation.

\item Contract support should not introduce unnecessarily new undefined behaviors
into the language.

\end{itemize}
 
For any checked mode, an implementation is required to perform the
checks specified by the associated mode. The behavior in response to a broken
contract should be implementation defined.

\subsubsection{Constraints on broken contracts behavior}

A \emph{checked mode} needs to define its behavior when a checked
contract is broken.

Some choices for an implementation could include:

\begin{itemize}

\item \textbf{Terminate the program}: The program would be terminated. However
it would still be allowed to set a \emph{terminate handler}.

\item \textbf{Call a handler function}: This is very similar to the previous
option but with an specific different handler function for broken contracts.

\item \textbf{Throw an exception}: A pre-defined exception could be thrown when
a contract is checked to be broken (e.g. \cppid{broken\_contract}). However,
this option could eventually prevent that any function with a contract could be
made \cppkey{noexcept}.

\end{itemize}

