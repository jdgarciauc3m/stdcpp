\subsection{Inline constructors proposal}

We present here the inline constructors proposal, which may be seen as more restrictive.

\subsubsection{Run-time bound array data members}

We propose a new syntax for introducing array data members of unspecified size.
The size will be specified at construction time.

\begin{lstlisting}
class my_vec {
public:
  // ...
private:
  int z;
  double[] v; //rtb array
  int[] w; // rtb array
};
\end{lstlisting}

While, the initial proposed syntax was the (perhaps more natural) 
\verb+T var[]+, it was noted that this might clash with C's empty array bounds.

Any class that has one or more run-time bound array data members becomes a
\emph{run-time size bound class}.

\subsubsection{Constructors}

A \emph{run-time size bound class} needs to specify the size of its array data
members upon construction.

Array sizes specifications need to be known from any constructor call site.
Thus, we require that a run-time size bound class has all its
constructors declared \verb+inline+. Any of these constructors needs  to specify
the size of each \emph{run-time bound array data member}. We propose the
following syntax:

\begin{lstlisting}
class my_vec {
public:
  my_vec(int n) : z{n}, v[n]{}, w[n*2]{} {/*...*/}
  my_vec() : z{0}, v[2]{}, w[4]{} {/*...*/}
  // ...
private
  int z;
  double[] v;
  int[] w;
};
\end{lstlisting}

Another option suggested by Daveed and collected in N3810 was describing
underlying storage within constructor declaration. This is illustrated in the
following example taken from N3810:

\begin{lstlisting}
struct MyArray {
  MyArray(int n) double storage[n];
  int size() const { return s; }
private:
  MyPtr<double> p;
  int s;
};

MyArray::MyArray(int n): s(n), p(storage) {}
\end{lstlisting}

We think that the major problem of such syntax is the case where a class needs
to have more than one RTB data member. We recognize that this is not a frequent
case.  Besides, we think that this suggestion makes code more difficult to read
by users.

It has also been suggested to simplify the syntax of RTB array data members to
the following:

\begin{lstlisting}
class my_vec {
public:
  my_vec(int n) : z{n}, v{n}, w{n*2} {/*...*/}
  my_vec() : z{0}, v{2}, w{4} {/*...*/}
  // ...
private
  int z;
  double[] v;
  int[] w;
};
\end{lstlisting}

While this syntax seems simpler, it may be considered confusing that the
initializer in this case is the size of the array and not a value. Besides, it
also prevents a user of potentially initializing a small array during
construction.

\begin{lstlisting}
class vec {
public:
  vec() : v[2]{1.0, 2.0} {/*...*/}
  // ...
private
  double[] v;
};
\end{lstlisting}

\subsubsection{Implementing \emph{bs\_array}}

One of the advantages of \emph{run-time size bound classes} is that they allow a
really \emph{pure-library} implementation of \verb+bs_array+.

\begin{lstlisting}
template <class T>
class bs_array {
public:
  using value_type = T;

  bs_array(int n) : sz{n}, v[n]{} {}
  bs_array(const bs_array & a) = default;
  bs_array(bs_array &&) = delete;

  // Unchecked access
  T & operator[](int i) { return v[i]; }
  const T & operator[](int i)  const { return v[i]; }

  // Range checked access
  T & at(int i);
  const T & at(int i) cosnt;

  // begin/end
  T * begin() { return v; }
  const T * begin() const { return v; }
  T * end() { return v+sz; }
  const T * end() { return v+sz; }

  int size() const { return sz; }
  T * data() { return v; }

private:
  int sz;
  T[] v;
};
\end{lstlisting}

We propose that a class like this is standardized. While we are not proposing a
specific name for such class, alternate names could be \verb+auto_array+,
\verb+stack_array+, \verb+dynarray+ or even \verb+autodynarray+.


\subsubsection{Initialization options}

The syntax needs to support several options for initialization as a balance
needs to be made between performance and flexibility. For this reason, we have
kept separate the specification of the data member size and its value in the
constructor.

If no values are provided in the initializer for the array data member, each
element in the array is \emph{value-initialized}.

\begin{lstlisting}
class vec {
public:
  vec(int n) : 
    v[n]{}, // v[i] = 0.0
    w[n]{}  // w[i] = point{}
  {}
private:
  double[] v;
  point[] w;
};
\end{lstlisting}

If braces are not present after the size specification, each element is
\emph{default-initialized} (including no initialization for scalars):


\begin{lstlisting}
class vec {
public:
  vec(int n) : 
    v[n], // v[i] is not initialized
    w[n]  // w[i] = point{}
  {}
private:
  double[] v;
  point[] w;
};
\end{lstlisting}

It is also possible to provide an \emph{initalizer-list} in the data member initializer:

\begin{lstlisting}
class vec {
public:
  vec() : 
    v[2]{-1.0, 2.5}, 
    w[2]{ {1.0,1.0}, {2.5,2.5} }
  {}
private:
  double[] v;
  point[] w;
};
\end{lstlisting}

If the \emph{initializer-list} provides less initializers than the array data member
size, the rest of elements are value initialized. 

However, if the number of
initializers exceed the number of elements, exceeding initializers are
discarded. Note that this condition cannot be checked until run time making any
compile time diagnostic not viable. An alternate solution here, could be
throwing an exception, but we are not currently proposing this alternative.

\begin{lstlisting}
class vec {
public:
  vec(int n) :
    v[n]{1.0, 2.0, 3.0}
  {}
private:
  double[] v;
};

vec a{4}; // a == {1.0, 2.0, 3.0, 0.0}
vec b{2}; // b == {1.0, 2.0}
\end{lstlisting}

\subsubsection{Run-time size bound objects as data members}

If a data member of a class is of a type that is a \emph{run-time size bound class}, the the new 
class is also a \emph{run-time size bound class} and the same restrictions are
applied to it. In particular, its constructor also needs to be defined inline.

\begin{lstlisting}
class A {
public:
  A(int n) : v[n]{} {}
private:
  double[] v;
};

class B {
public:
  B() : a{4} {}
private:
  A a;
};
\end{lstlisting}

The same applies to a inheritance.

\begin{lstlisting}
class A {
public:
  A(int n) : v[n]{} {}
private:
  double[] v;
};

class B : public A {
public:
  B() : A{4} {}
};
\end{lstlisting}

\subsubsection{Contexts of use}

We propose that the only use of a \emph{run-time size bound class} object is as
an automatic variable. In particular, we propose to ban any use implying dynamic
memory.

\begin{lstlisting}
class A {
public:
  A(int n) : v[n]{} {}
private:
  double[] v;
};

void f() {
  A a{4}; // OK;
  A * p = new A{4}; // Error
  vector<A> w; // Error
}
\end{lstlisting}

\subsubsection{\emph{sizeof}}

Operator \verb+sizeof+ is not supported on any type or object of a \emph{run-time size bound class}.

\subsubsection{Non-inlined constructors}

Imposing the (transitive) requirement that constructors need to be inlined may
be seen as very restrictive. We think that the most common use of this feature
is defining automatic storage objects at function scope level. Thus, this
restriction may be not so critical.

However, an alternate approach could be allowing that a constructor body is
first inlined and then its body is refined.

\begin{lstlisting}
class vec {
public:
  vec(int n) : v[n]{} {} // inline definition
private:
  double[] v;
};

// Definition somehwere else
vec::vec(int n) // Member initiation not allowed here
{
  do_some_additional_thing();
}
\end{lstlisting}

In this case the inline definition can be used for computing the size of the
object at the construction call site. However, after that the constructor body
(if one is existing) may be called. For the sake of avoiding redundancy in this
case, we do not allow to repeat the member initiation list.

\subsubsection{Size determination}
\label{sec:inline-size-det}

The current proposal implies that the size of any \emph{run-time bound array
data member} can be derived from the environment where the inline constructor is
defined.

In particular, this proposal does not allow the following example (slightly
modified from an example provided by Lawrence Crowl).

\begin{lstlisting}
// a.h
struct A {
  bs_array<double> storage;
  A(int n);
};

// a.cc
  extern int config;
  A::A(int n) : storage(n*config) {}
\end{lstlisting}

As \verb+struct A+ has a data member which is a \emph{run-time size bound class}
(the \verb+bs_array+), it is considered itself to be also a \emph{run-time size
bound class} and the same restrictions apply to it. Thus, it needs that its
constructor is defined inline.

In this proposal we have not addressed this problem. We think it should be first
clarified if this use case is important enough to be solved.

Possible solutions are:

\begin{itemize}

\item Require that the variable (i.e. \verb+config+) is visible at the point of definition of the
inline constructor.

\item Modify the proposal so that it is not required that the constructor is
defined inline.

\end{itemize}

\subsubsection{Multi-dimensional data members}

This proposal initially tries to support single-dimension \emph{run-time bound}
data members. However, if needed it could be extended to support multiple
dimension \emph{run-time bound} data members.

\begin{lstlisting}
class stack_matrix {
public:
  stack_matrix(int r, int c) : v[r][c]{} {}
  // ...
private:
  double[][] v;
};
\end{lstlisting}

