\section{Problem}

\begin{frame}{The problem with VLA's}
  \begin{itemize}
    \item Variable Length Arrays in C since C99.
      \begin{itemize}
        \item Problematic in C++.
      \end{itemize}
    \item Past proposals:
      \begin{itemize}
        \item Runtime-sized arrays with automatic storage.
          \begin{itemize}
            \item N3366, N3412, N3467, N3497, N3639.
          \end{itemize}
        \item \texttt{dynarray}.
          \begin{itemize}
            \item N3532, N3662
          \end{itemize}
        \item \texttt{bs\_array}.
          \begin{itemize}
            \item N3810
          \end{itemize}
      \end{itemize}
    \item Alternatives sumarized in N3810.
  \end{itemize}
\end{frame}

\subsection{Runtime-sized arrays with automatic storage}

\begin{frame}[fragile]{Use case}
  \begin{itemize}
    \item A user wishes to allocate a local array whose size is not known at compile-time, but at runtime only.
  \end{itemize}
\pause
\begin{lstlisting}[style=cpp11]
std::string f(int n) {
  std::vector<int> v(n); // std::vector allocation
  std::iota(v.begin(), v.end(), 1);
  std::string result;
  for (auto x : v) {
    result += std::to_string(x);
  }
  return result;
} // std::vector deallocation
\end{lstlisting}
  \begin{itemize}
    \item \alert{Use of memory allocation leads to performance penalty}.
  \end{itemize}
\end{frame}

\begin{frame}[t,fragile]{Avoiding dynamic memory allocation}
  \begin{itemize}
    \item \emph{Use a runtime sized array with automatic storage}.
  \end{itemize}
\begin{lstlisting}[style=cpp11]
std::string f(int n) {
  int v[n]; // run-time sized array with automatic storage
  std::iota(v, v + n, 1);
  std::string result;
  for (int i=0; i<n; ++i) {
    result += std::to_string(x);
  }
  return result;
} 
\end{lstlisting}
\end{frame}

\begin{frame}[t,fragile]{Promoting pointer-plus-size interface}
\begin{lstlisting}[style=cpp11]
std::string g(int * v, std::size_t n) {
  std::string result;
  for (std::size_t i=0; i!=n; ++i) {
    result += std::to_string(v[i]);
  }
  return result;
}

std::string f(int n) {
  int v[n];
  std::iota(v, v + n, 1);
  return g(v, n);
} 
\end{lstlisting}
\pause
\begin{itemize}
  \item Why this interface is not desirable?
    \begin{itemize}
      \item Number of elements not available $\Rightarrow$ User needs to \emph{remember} size.
      \item Type to access elements can implicitly changes after array decayed to pointer.
    \end{itemize}
\end{itemize}
\end{frame}

\subsection{\texttt{dynarray} class}

\begin{frame}{\texttt{dynarray}}
  \begin{itemize}
    \item A class with compiler support to eventually allocate on stack.
      \begin{itemize}
        \item A pure-library implementation is possible.
        \item But it would probably be based on dynamic memory allocation.
        \item Stack allocation would require \emph{compiler magic}.
      \end{itemize}
    \item Pro:
      \begin{itemize}
        \item It solves the \emph{interface problem}.
      \end{itemize}
    \item Con:
      \begin{itemize}
        \item \texttt{dynarray} evolved to \emph{almost a container}.
        \item No guarantee whether an object will be allocated in automatic or dynamic storage.
      \end{itemize}
  \end{itemize}
\end{frame}

\begin{frame}[t,fragile]{Using \texttt{dynarray}}
\begin{lstlisting}[style=cpp11]
std::string g(const dynarray<int> & v) {
  std::string result;
  for (auto x : v) {
    result += std::to_string(x);
  }
  return result;
}

std::string f(int n) {
  dynarray<int> v{n}; // Where is this allocated?
  std::iota(v.begin(), v.end(), 1);
  return g(v);
} 
\end{lstlisting}
\end{frame}

\subsection{\texttt{bs\_array} class}

\begin{frame}[t,fragile]{A \emph{minimal stack-allocated array}}
\begin{lstlisting}[style=cpp11]
template<class T>
class bs_array { // basic stack-allocated array
public:
  using value_type = T;

  bs_array(int n); // n elements
  T& operator[](int i); 
  const T& operator(int i) const; 
  T& at(int i); 
  const T& at(int i) const; 
  T* begin() { return a; } 
  const T* begin() const { return a; }
  T* end() { return a+n; }
  const T* end() const { return a+n; }
  int size();
  T* data();
private:
  T a[n]; // for exposition only, a is stack allocated
};
\end{lstlisting}
\end{frame}

\begin{frame}[t,fragile]{Using \texttt{bs\_array}}
\begin{lstlisting}[style=cpp11]
std::string g(const bs_array<int> & v) {
  std::string result;
  for (auto x : v) {
    result += std::to_string(x);
  }
  return result;
}

std::string f(int n) {
  bs_array<int> v{n}; // allocated on the stack
  std::iota(v.begin(), v.end(), 1);
  return g(v);
}
\end{lstlisting}
\begin{itemize}
  \item It always allocates on the stack.
  \item No wrong interface promotion.
  \item \alert{Needs compiler support}.
\end{itemize}
\end{frame}
