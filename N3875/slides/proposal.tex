\section{Proposal}

\subsection{Run-time size bound array}

\begin{frame}[t]{Basic idea}
\begin{itemize}[<+->]
  \item Core idea:
    \begin{itemize}
      \item Allow a class to have one or more run-time bound array data members.
        \begin{itemize}
          \item Array data members of unspecified number of elements.
          \item Constructor responsible for setting the size upon object construction.
        \end{itemize}
    \end{itemize}
  \item Alternatives:
    \begin{itemize}
      \item Inline constructors.
        \begin{itemize}
          \item Constructors required to be inline.
          \item Array data members size set in constructor initilization list.
        \end{itemize}
      \item Sized constructos.
        \begin{itemize}
          \item Array data members size is part of constructor declaration.
        \end{itemize}
    \end{itemize}
\end{itemize}
\end{frame}

\begin{frame}[t,fragile]{Runtime bound array data members}
  \begin{itemize}
    \item New syntax for array data members of unspecified size.
  \end{itemize}
\pause
\begin{lstlisting}[style=cpp11]
class my_vec {
public:
  // ...
private:
  int z;
  double[] v; //rtb array
  int[] w; // rtb array
};
\end{lstlisting}
  \begin{itemize}
    \item Alternate syntax:
  \end{itemize}
\begin{lstlisting}[style=cpp11]
T var[]; // May clash with C?
\end{lstlisting}
\end{frame}

\subsection{Inline constructors}

\begin{frame}[t,fragile]{Constructors}
\begin{itemize}
  \item Needs to specify size of array data members upon construction.
  \item Require constructors to be inline.
  \item Specify array size extending initialization-list syntax.
\end{itemize}
\begin{lstlisting}[style=cpp11]
class my_vec {
public:
  my_vec(int n) : z{n}, v[n]{}, w[n*2]{} {/*...*/}
  my_vec() : z{0}, v[2]{}, w[4]{} {/*...*/}
  // ...
private
  int z;
  double[] v;
  int[] w;
};
\end{lstlisting}
\end{frame}

\begin{frame}[t,fragile]{Implementing \texttt{bs\_array} (I)}
\begin{lstlisting}[style=cpp11]
template <class T>
class bs_array {
public:
  using value_type = T;

  bs_array(int n) : sz{n}, v[n]{} {}
  bs_array(const bs_array & a) = default;
  bs_array(bs_array &&) = delete;

  // Unchecked access
  T & operator[](int i) { return v[i]; }
  const T & operator[](int i)  const { return v[i]; }

  // Range checked access
  T & at(int i);
  const T & at(int i) const;

  // ...
\end{lstlisting}
\end{frame}

\begin{frame}[t,fragile]{Implementing \texttt{bs\_array} (and II)}
\begin{lstlisting}[style=cpp11]
  // ...

  // begin/end
  T * begin() { return v; }
  const T * begin() const { return v; }
  T * end() { return v+sz; }
  const T * end() { return v+sz; }

  int size() const { return sz; }
  T * data() { return v; }

private:
  int sz;
  T[] v;
};
\end{lstlisting}
\end{frame}

\begin{frame}[t,fragile]{Initialization options (I)}
  \begin{itemize}
    \item Separate specification of array size and initial values.
    \item If no values are provided, each element is \emph{value-initialized}.
  \end{itemize}
\begin{lstlisting}[style=cpp11]
class vec {
public:
  vec(int n) : 
    v[n]{}, // v[i] = 0.0
    w[n]{}  // w[i] = point{}
  {}
private:
  double[] v;
  point[] w;
};
\end{lstlisting}
\end{frame}

\begin{frame}[t,fragile]{Initialization options (II)}
  \begin{itemize}
    \item If braces are not present, each element is \emph{default-initialized}.
  \end{itemize}
\begin{lstlisting}[style=cpp11]
class vec {
public:
  vec(int n) : 
    v[n], // v[i] is not initialized
    w[n]  // w[i] = point{}
  {}
private:
  double[] v;
  point[] w;
};
\end{lstlisting}
\end{frame}

\begin{frame}[t,fragile]{Initialization options (III)}
  \begin{itemize}
    \item An \emph{initializer-list} can be used initialize the array.
      \begin{itemize}
        \item If less elements than size are provided, the rest are \emph{value-initialized}.
        \item If more elements than size are provided, exceeding elements are discarded.
      \end{itemize}
  \end{itemize}
\begin{lstlisting}[style=cpp11]
class vec {
public:
  vec() : 
    v[2]{-1.0, 2.5}, 
    w[2]{ {1.0,1.0}, {2.5,2.5} }
  {}
private:
  double[] v;
  point[] w;
};
\end{lstlisting}
\end{frame}

\begin{frame}[t,fragile]{Using a RB class as data member}
  \begin{itemize}
    \item If a data member is of a runtime size bound class $\Rightarrow$ the new class is runtime size bound class.
  \end{itemize}
\begin{lstlisting}[style=cpp11]
class A {
public:
  A(int n) : v[n]{} {}
private:
  double[] v;
};

class B {
public:
  B() : a{4} {}
private:
  A a;
};
\end{lstlisting}
\end{frame}

\begin{frame}[t,fragile]{Inheriting from a RB class}
  \begin{itemize}
    \item If a base class is of a runtime size bound class $\Rightarrow$ the new class is runtime size bound class.
  \end{itemize}
\begin{lstlisting}[style=cpp11]
class A {
public:
  A(int n) : v[n]{} {}
private:
  double[] v;
};

class B : public A {
public:
  B() : A{4} {}
};
\end{lstlisting}
\end{frame}

\begin{frame}[t,fragile]{Contexts of use}
\begin{itemize}
  \item Only objects as automatic variables.
\end{itemize}
\begin{lstlisting}[style=cpp11]
class A {
public:
  A(int n) : v[n]{} {}
private:
  double[] v;
};

void f() {
  A a{4}; // OK;
  A * p = new A{4}; // Error
  vector<A> w; // Error
}
\end{lstlisting}
\end{frame}

\begin{frame}[t]{Size}
  \begin{itemize}
    \item Operator \cppkey{sizeof} is not supported for any type of object of run-time size bound class.
  \end{itemize}
\end{frame}

\begin{frame}[t,fragile]{Non-inlined constructors}
  \begin{itemize}
    \item Transitive requirement of inline constructors seen too restrictive.
      \begin{itemize}
        \item Very common case: Automatic objects at function scope.
        \item Other case: Composition in other classes (abstraction).
          \begin{itemize}
            \item Restriction may become problematic here.
          \end{itemize}
      \end{itemize}
    \pause
    \item Alternate approach:
      \begin{itemize}
        \item Allow inlining and then body refinement.
      \end{itemize}
  \end{itemize}
\begin{lstlisting}[style=cpp11]
class vec {
public:
  vec(int n) : v[n]{} {} // inline definition
private:
  double[] v;
};
// Definition somehwere else. No init-list allowed
vec::vec(int n) {
  do_some_additional_thing();
}
\end{lstlisting}
\end{frame}

\begin{frame}[t,fragile]{Multi-dimensional data members}
\begin{lstlisting}[style=cpp11]
class stack_matrix {
public:
  stack_matrix(int r, int c) : v[r][c]{} {}
  // ...
private:
  double[][] v;
};
\end{lstlisting}
\end{frame}

\subsection{Sized constructors}

\begin{frame}[t,fragile]{Size specifications}
  \begin{itemize}
    \item Add a \emph{sizeof specifier} in constructor declaration.
      \begin{itemize}
        \item Only in declaration.
      \end{itemize}
  \end{itemize}
\begin{lstlisting}[style=cpp11]
class my_vec {
public:
  my_vec(int n) sizeof(v[n], w[2*n]);
  my_vec() sizeof(v[2], w[4]);
private:
  int z;
  double[] v;
  int[] w;
};

my_vec::my_vec(int n) : z{n} {}

my_vec::my_vec() : z{0} {}
\end{lstlisting}
\end{frame}


\begin{frame}[t,fragile]{Implementing \texttt{bs\_array} (I)}
\begin{lstlisting}[style=cpp11]
class bs_array {
public:
  using value type = T;

  bs_array(int n) sizeof(v[n]);
  bs array(const bs array & a) = default;
  bs array(bs array &&) = delete;

  // Unchecked access
  T & operator[](int i) { return v[i]; }
  const T & operator[](int i) const { return v[i]; }

  // Range checked access
  T & at(int i );
  const T & at(int i) cosnt;

  // ...
\end{lstlisting}
\end{frame}

\begin{frame}[t,fragile]{Implementing \texttt{bs\_array} (and II)}
\begin{lstlisting}[style=cpp11]
  // ...

  // begin/end
  T * begin() { return v; }
  const T * begin() const { return v; }
  T * end() { return v+sz; }
  const T * end() { return v+sz; }

  int size () const { return sz; }
  T * data() { return v; }

private:
  int sz;
  T[] v;
};

\end{lstlisting}
\end{frame}
