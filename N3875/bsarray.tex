\subsection{\emph{bs\_array}}

In N3810~\cite{n3810} a minimal \emph{basic stack-allocated} array is proposed.
The idea is that this class relates to an array with dynamic extent as
\verb+std::array+ relates to an array with static extent.

Here is a possible outline taken from N3810.

\begin{lstlisting}
template<class T>
class bs_array { // basic stack-allocated array
  using value_type = T;

  bs_array(int n); // n elements
  // default copy, no move, maybe copy to vector/other_containers

  T& operator[](int i); 
  const T& operator(int i) const; 
  T& at(int i); 
  const T& at(int i) const; 

  T* begin() { return a; } 
  const T* begin() const { return a; }
  T* end() { return a+n; }
  const T* end() const { return a+n; }

  int size();

  T* data();

private:
  T a[n]; // for exposition only, a is stack allocated
};
\end{lstlisting}

This approach addresses our concern about allocation as \verb+bs_array+ is
guaranteed to be allocated on the stack. Now, the interface promotion problem is
solved at the same time that storage allocation is predictable.

\begin{lstlisting}
std::string g(const bs_array<int> & v) {
  std::string result;
  for (auto x : v) {
    result += std::to_string(x);
  }
  return result;
}

std::string f(int n) {
  bs_array<int> v{n}; // allocated on the stack
  std::iota(v.begin(), v.end(), 1);
  return g(v);
}
\end{lstlisting}



However, \verb+bs_array+ shares with \verb+dynarray+ the need for compiler
support. \textbf{While this is acceptable, we beleive that a more general
solution can be achieved at the language level}.
