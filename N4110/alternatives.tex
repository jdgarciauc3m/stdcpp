\section{Alternatives}

Mechanisms have been proposed in the past to handle correctness both in the
library and the language level.

\subsection{Library solutions}

Many solutions have been proposed for supporting contracts in C++.

\subsubsection{The \texttt{assert} macro}

The most classical solution has been extensive use of \texttt{assert} macro from
C. The macro expands to a call to \texttt{abort()} when macro \texttt{NDEBUG} is
not defined. Otherwise it expands to nothing. This leads to a set of issues:

\begin{itemize}

\item Assertions are evaluated in the callee site. This means that decision
about checking is taken when the callee is compiled. Thus, one can have either a
fully checked library with a performance penalty or a faster but unchecked
library.

\item Assertions cannot be distinguished from regular code. This makes impossible
for a compiler to take advantage from semantic information provided in the form
of a contract. This information could eventually be used for optimizations or
for supporting formal or semi-formal verifications.

\item Assertions do not allow to distinguish between different kinds of
conditions. Making these distinctions is important if they are going to be
activated/deactivated separately.

\end{itemize}

\subsubsection{Centralized defensive-programming library}

The most recent proposal \cite{n3997} in the C++ standards committee about contract programming suggested a
library solution for standardization. The proposal contained:

\begin{itemize}

\item A definition of a set of build modes (\emph{safe-build}, \emph{debug},
\emph{optimized}).

\item A configuration violation-handler mechanism to allow the application
developer to make its choice at run-time among several options (including
aborting the program, throwing an exception, \ldots).

\item A set of macros for expressing assertions inside implementations of
functions.

\end{itemize}

\subsubsection{Limitations of library solutions}

While this approach definitely presents key concepts in terms of contract
programming, it seems to be focused on preconditions and does not clarify a
route for postconditions and/or invariants.

A key point is the fact that a library solution effectively makes the contract
part of the implementation. In contrast, a contract should be seen as part of
component specification. That approach would also allow that different clients
may use the component with or without contract checking.

The introduction of a library solution for contract checking would also prevent
a later language-based solution. A language solution may provide opportunities
for better development tools. In particular:

\begin{itemize}

\item Language-based solution may be a corner-stone to support formal
verification tools. Experiences with \emph{SPARK}
(\url{http://www.spark-2014.org/}) have shown that a language-based contract
mechanism is essential for formal verification.

\item A compiler may use the contract information for better code generation and
take advantage of the extra information for code optimization.

\end{itemize}

\subsection{Language solutions}

In 2005 a proposal to add contract programming was considered by the Evolution
Working Group. The initial proposal~\cite{n1613} contained a discussion on reasons to prefer
a language solution over a library solution. Among others, the following were
included:

\begin{itemize}

\item Side effects cannot be detected by library solutions.

\item Assertions about return value or about side-effects cannot be done without manually copying the
result.

\item There is no clear mechanism for inheriting assertions in public virtual
functions.

\end{itemize}

Proposal~\cite{n1773} was discussed at the Lillehammer meeting.
Minutes are rather short:

\vspace{12pt}
\noindent
\framebox[\textwidth][l]{\parbox{.99\textwidth}{
(discussed) N1773 05-0033 : Proposal to add Contract Programming to C++ (revision 2) L. Crowl, T. Ottosen

Straw poll: Evaluate synergies (7/4/3/0) 

(Means: authors get 1 hour floor time on the next meeting to present their findings)
}}
\vspace{12pt}

At Mont Tremblant meeting a revised proposal~\cite{n1866} was discussed. However
no consensus could be achieved on whether C++0x should pursue contract
programming.

The proposal included static assertions and runtime assertions. In this paper
the focus is on runtime assertions, as static assertions can be handled by other
means in the language. The basic approach included the ability to add
preconditions and postconditions to functions and invariants to classes. A
simplified example of its use is the following:

\begin{lstlisting}[style=cppcontract05]
double sqrt( double r )
    precondition
    {
        r > 0.;
    }
    postcondition( result )
    {
        equal_within_precision( result * result, r );
    }
\end{lstlisting}

Another interesting feature was the use of \texttt{if} statements to express
implication assertions:

\begin{lstlisting}[style=cppcontract05]
class vector {
  // ...
  iterator begin()
    postcondition(result) { if (empty()) result==end(); }
  // ...
};
\end{lstlisting}

In this set of proposals the default behavior when a contract is violated is to
call \texttt{terminate()}. However, a set of functions are provided to change
this default behavior by setting a handler:

\begin{lstlisting}
typedef void (*broken_contract_handler)();
    
broken_contract_handler set_precondition_broken_handler(broken_contract_handler r) throw();
broken_contract_handler set_postcondition_broken_handler(broken_contract_handler r) throw();
broken_contract_handler set_class_invariant_broken_handler(broken_contract_handler r) throw();
broken_contract_handler set_namespace_invariant_broken_handler(broken_contract_handler r) throw();
\end{lstlisting}

This model also included specific rules for handling the cases from inheritance
and virtual functions:

\begin{enumerate}

\item Only the first declaration/definition of F can have a precondition.

\item Pure virtual functions can have contracts.

\item Postconditions from a function in a base class and the overridden version
in a derived class are \emph{AND}ed together.

\item If \emph{F} overrides more than one virtual function due to multiple
inheritance, the precondition on \emph{F} is an \emph{OR}ing of the
preconditions of all the functions \emph{F} override and the postcondition on
\emph{F} is an \emph{AND}ing of the postconditions of all the functions \emph{F}
override and \emph{F}'s own postcondition.  \end{enumerate}

The implementation model proposed implied that all the code was generated at the
callee site. This approach allowed to address code bloat issues. For every
function with a contract the compiler would generate three functions:

\begin{enumerate}

\item \textbf{Core function}: This is the function without any checking at all.

\item \textbf{Postcondition evaluator}: This one calls the core function and
then executes the postcondition checks.

\item \textbf{Precondition evaluator}: This one executes de precondition checks
and the calls the postcondition evaluator.

\end{enumerate}

Still this model supports the concepts of multiple build modes, where
postconditions can be activated/deactivated when compiling the callee and
preconditions can be activated/deactivated when compiling the caller.

This model requires to standardize multiple build modes which would be something
new to the C++ standard. Besides, it prevents any attempt to optimize out
unneeded checks.



\subsection{Other languages with contract support}

\subsubsection{Eiffel}

A classical reference in contract programming is the Eiffel
language~\cite{iso-eiffel,ecma-eiffel}.

A typical contract example in Eiffel is as follows:

\begin{lstlisting}[language=Eiffel]
class DICTIONARY [ELEMENT]
feature
	put (x: ELEMENT; key: STRING) is
		require
			count <= capacity
			not key.empty 
		ensure
			has (x)
			item (key) = x
			count = old count + 1
		end

	... Interface specifications of other features ...

invariant
	0 <= count
	count <= capacity
end
\end{lstlisting}

An Eiffel implementation (section 8.8.29 of \cite{ecma-eiffel}) must provide
facilities for enabling and disabling assertion monitoring. It allows a variety
of methods for setting the assertion monitoring level either statically (at
compile time) or dynamically (at runtime). This can be done through control
information in the Eiffel text or through outside elements such as a user
interface or a configuration file.

Eiffel includes preconditions, postconditions, class invariants and loop
invariants. check instructions and loop variants. Each of this check can be
activated separately either for specific classes, specific \emph{clusters} or
the entire system.

At least an implementation must support the following for modes:

\begin{enumerate}

\item Statically disable all monitoring for the entire system.

\item Statically enable precondition monitoring for an entire system.

\item Statically enable precondition monitoring for specified classes.

\item Statically enable all assertion monitoring for an entire system.

\end{enumerate}

If an assertion is violated an exception (\texttt{ASSERTION\_VIOLATION}) is
raised to notify that violation.

\subsubsection{Ada 2012}

Ada~\cite{Ada2012} has recently incorporated contract programming that was
successfully used previously in SPARK. The approach is rather similar to Eiffel.

\begin{lstlisting}[language=Ada]
procedure Push (S: in out Stack; E: in Element)
 with Pre => Not_Full (S),
     Post => (S.Length = S'Old.Length + 1) and
             (S.Data (S.Length) = E) and
             (for all J in 1 .. S'Old.Length => 
                S.Data (J) = S'Old.Data (J)); 
\end{lstlisting}

If an assertion is violated an exception is raised to notify that violation.

\subsubsection{D}

The D programming language (\url{http://dlang.org}) also provides a contract
mechanism.

\begin{lstlisting}[style=dlang]
long square_root(long x)
  in
  {
    assert(x >= 0);
  }
  out (result)
  {
    assert((result * result) <= x && (result+1) * (result+1) > x);
  }
  body
  {
    return cast(long)std.math.sqrt(cast(real)x);
  }
\end{lstlisting}

It also allows the definition of invariants:

\begin{lstlisting}[style=dlang]
class Date {
  int day;
  int hour;

  invariant // start class invariant
  {
    assert( 1 <= day && day <= 31 );
    assert( 0 <= hour && hour < 24 );
  }
}
\end{lstlisting}

In contrast with other options, D allows any arbitrary statement to appear in a
precondition, postcondition or invariant making necessary to use the
\texttt{assert} statement. This also makes possible that contracts may exhibit
side effects.

Again, when a violation occurs an error is thrown. However, in release mode all
checks are removed.
