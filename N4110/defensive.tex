\section{Addressing centralized defensive programming}

This section tries to explore how the presented solution would be addressing the
issues presented in \cite{n3997}.

\subsection{The wide versus narrow contracts debate}

The importance of the distinction between wide and narrow contracts must be
acknowledged. However, both establish a contract. It is my view that the
difference lies in the fact of how a violation is handled.

A \emph{wide contract} only has strong preconditions. Those preconditions are never
removed (regardless the checking mode) and the always have the same effect they
throw an exception.

\begin{lstlisting}
const T & std::vector::at(size_t index) const
  expects<out_of_range>(index<size());
\end{lstlisting}

A \emph{narrow contract} may have weak preconditions. Those preconditions may be
checked or not, depending on the context and compilation mode.

\begin{lstlisting}
const T & std::vector::at(size_t index) const
  expects(index<size());
\end{lstlisting}

One of the reasons that initiated the wide versus narrow contract debate was to
establish limitations to the aggressive use on \texttt{noexcept} in the standard
library specification. It is worth to note that the approach taking in this
paper that the contract is part of the interface and not of the implementation
would eventually allow that functions with contracts (either narrow or wide)
would still be able to be marked \texttt{noexcept}.

\begin{lstlisting}
const T & std::vector::at(size_t index) const noexcept
  expects<out_of_range>(index<size());
const T & std::vector::at(size_t index) const noexcept
  expects(index<size());
\end{lstlisting}

\subsection{Artificially widening of contracts}

Expressing a weak contract is not a way of widening that contract.

Given the following example from~\cite{n3997}:

\begin{lstlisting}
class date {
 // ...
 public:
 // ... 
 void set_ymd(int year, int month, int day); // narrow
 // Set the value of this object to the specified
 // 'year', 'month', and 'day'. The behavior is 
 // undefined unless 'year', 'month', and 'day' 
 // together represent a valid/supported date value. 
\end{lstlisting}

We would express it:

\begin{lstlisting}
class date {
  public:
    void set_ymd(int y, int m, int d)
      expects(valid_date(y,m,d));
    //...
  private:
    static bool valid_date(int y, int m, int d);
      // Tell me if it represents a valid/supported date value
\end{lstlisting}

This rewriting does not change the fact that the contract is narrow. Moreover,
nothing stops the writer to mark the function as \texttt{noexcept}.

The reasons for not artificially widening narrow contracts (runtime cost,
development cost, code size, extensibility, and defensive programming) are all
valid and the guideline of not widening otherwise narrow contracts is correct.

\cppquestion{Explicitly expressing contracts as part of the interface does not
widens an otherwise narrow contract}

\subsection{Contracts and high-level requirements}

A contract solution must satisfy high-level requirements that are different for
the library developer and and application developer.

A library developer must be able to:

\begin{itemize}

\item Easily implement defensive checks which are active in appropriate
defensive build modes. The proposed solution easily achieves this goal as
contracts can be activated, although this is not the default mode.

\item Easily check that defensive checks are working as intended. 

\end{itemize}

An application owner must be able to:

\begin{itemize}

\item Coarsely specify (at compile time) the overall runtime validation
overhead.

\item Specify precisely (at runtime) the action to take if an error is detected. 

\item Link translation units compiled with different levels of runtime validation. 

\end{itemize}

This goals are achieved as it is the client the one who makes the choice on
whether checks are activated or deactivated.

\subsection{Examples}

\subsubsection{Contract checking}

\begin{lstlisting}
std::size_t other_strlen(onst char * str) 
{
  CONTRACT_ASSERT(str);
  // return length
}
\end{lstlisting}

This example can be easily expressed as:

\begin{lstlisting}
std::size_t other_strlen(onst char * str) 
  expects(str!=nullptr)
{
  // return length
}
\end{lstlisting}

\subsubsection{Safe contract checking}

\begin{lstlisting}
int get_int_array_element(int *array, int length, int index) 
{ 
 CONTRACT_ASSERT_SAFE(0 <= index ); 
 CONTRACT_ASSERT_SAFE(index < length); 

 return array[index]; 
} 
\end{lstlisting}

If the distinction between a contract and a \emph{safe} contract is needed,
\emph{safe} versions of contracts could be added.

\begin{lstlisting}
int get_int_array_element(int *array, int length, int index) 
  expects_safe(
    0<=index &&
    index < length)
{ 
 return array[index]; 
} 
\end{lstlisting}

However, this mode could easily lead to may developers defaulting to this mode
and resulting in over-checking. In any case, moving the checks to the interface
could allow to avoid the checks if they can be proved unneeded.

\subsubsection{Violation handling}

Setting a contract violation handler is supported by this paper and no
significant deviation from N3997 is seen.

\subsubsection{Testing contracts}

N3997 also provides a mechanism for testing that a function correctly tests
against its contract. This is done via macros 
\texttt{TEST\_CONTRACT\_ASSERT\_PASS} and
\texttt{TEST\_CONTRACT\_ASSERT\_FAIL}.
This feature is not currently addressed in this proposal. However, mechanisms
could be envisioned if this is really a needed feature. 
