\section{Introduction}

In the design of any library there is a tension between two issues that can be
seen as related: correctness and robustness.  \emph{Correctness} can be seen as
the degree to which a software component matches its specification.
\emph{Robustness} is the ability of a software component to react appropriately
to abnormal conditions.

Currently the C++ language and many libraries use a single feature for managing both
properties: exception handling. 

When a failure happens we use exceptions as an error reporting mechanism, to
notify that an error has occurred and needs to be handled somewhere else. In
this way, we allow decoupling the error identification from the error handling.

Besides, when a library detects that some assumed condition has not been met, it
needs a mechanism to react properly. We call this situation a \emph{contract
violation}. Some libraries just decide not to check contract violations and just
document those options. In that case (e.g. the C++ Standard Library), a contract
violation leads to undefined behavior.

In this paper, I claim that correctness and robustness are orthogonal properties
of a program and that they should be managed independently. This idea is not new, and
the approach of independent management of correctness and robustness can be
found in other somehow similar languages (e.g.  Ada, Eiffel, D).

\subsection{Robustness in the C++ standard library}

\emph{Robustness} is related to the identification and handling of abnormal
situations. Normally, these errors occur even if a program is completely
correct. A typical example is a program that cannot read from a file. Such a
failure is treated by throwing an exception an catching it in a different
context. Another example is the failure to allocate dynamic memory.

Currently, the C++ standard library identifies the cases when an abnormal
situation may happen, by specifying i) the condition that may cause that situation
and ii) the exception that will be thrown to notify of that situation.

For example, the standard allocator \emph{std::allocator} has the following \emph{Throws:}
clause in the specification of its \emph{allocate} member function:

\vspace{1em}
\noindent
\framebox[\textwidth][l]{
\emph{Throws}: \texttt{bad\_alloc} if the storage cannot be obtained.
}
\vspace{1em}

Even if a program is correct, it may fail in allocating memory.

\subsection{Correctness in the C++ standard library}

\emph{Correctness} is related to finding programming errors. Those situations happen
due to incorrectly programmed software. A correctly stated precondition
violation happens because the caller is not fulfilling that precondition before
performing the call. A correctly stated postcondition happens because the callee
is not fulfilling the postcondition after execution. A program failure (a
robustness error) cannot be avoided in a program, as it usually depends
on some external condition. In contrast, a contract violation (a correctness
error), should never happen in a correctly written program.

In its current form, the C++ standard library documents the preconditions of a function.
Section 17.4.6.11 states:

\vspace{1em}
\noindent
\framebox[\textwidth][l]{\parbox{.99\textwidth}{
\emph{Violation of the preconditions specified in a function’s Requires:
paragraph results in undefined behavior unless the function’s Throws: paragraph
specifies throwing an exception when the precondition is violated.}
}}
\vspace{1em}

Thus, in practice, there are two approaches for contract violations in the
standard library: either a precondition violation may result in undefined behavior
or it may be notified by throwing an exception.

\subsection{The narrow and wide contracts}

During C++11 standardization final stage there were arguments against aggressive
use of \textbf{noexcept}~\cite{n3248}. The key argument was that the fundamental
implication was that such approach would make \emph{``functions marked
\texttt{noexcept} difficult to test''}.  To overcome this difficulty two
approaches were outlined: either to lift the requirement that a
\texttt{noexcept} violation would result in a call to \texttt{terminate} or to
impose a severe criteria on when a standard library function can be marked
\texttt{noexcept}.

The route taken by C++11 was the definition of those criteria~\cite{n3279} based
on the definitions of \emph{wide contracts} (does not specify any undefined behavior
and has no precondition) and \emph{narrow contracts} (results in undefined
behavior when a precondition is violated). With those definitions the adopted guidelines
were:

\vspace{1em}
\framebox[\textwidth][l]{\parbox{.99\textwidth}{
\begin{itemize}

\item No library destructor should throw.

\item Each library function having a wide contract, that LWG agrees cannot throw, 
should be marked unconditionally \texttt{noexcept}.

\item If a library swap function, move-constructor, or move-assignment operator
is conditionally-wide, then it should be marked conditionally \texttt{noexcept}.

\item No other function should be marked conditionally \texttt{noexcept}.

\item Library functions designed for compatibility with ``C'' code, may be
marked as unconditionally noexcept.

\end{itemize}
}}
\vspace{1em}

There are two opposed views about the use of \texttt{noexcept}. On one hand,
there is the view that \texttt{noexcept} was designed to solve a very specific
problem to correctly handle throwing move constructors. This view tries to
limit \texttt{noexcept} to move operations. On the other hand, there is a
movement of marking \texttt{noexcept} as much functions as possible just
because they me lead to a performance benefit.

However, \texttt{noexcept} is a specification of an interface to indicate that a
a certain function does not throw exceptions at all. No less and no more than
that. Thus, those functions that do not throw to notify that an abnormal
situation has happened should be \texttt{noexcept}.

The current library guidelines make that functions that are conceptually
\texttt{noexcept} are not marked so. This may lead to some performance loses.
In fact, this may be seen to be against the principle of zero-overhead. However,
the key point is a different issue. Functions are not marked \texttt{noexcept}
even if they should not be throwing at all.
